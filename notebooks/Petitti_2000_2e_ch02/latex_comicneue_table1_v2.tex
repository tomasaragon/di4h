\documentclass{article}
\usepackage{amsmath}
\usepackage{booktabs}
\usepackage{fontspec}
\usepackage{unicode-math} % For setting math fonts
\setmainfont{Comic Neue} % Compile with lualatex 
\setmathfont{Latin Modern Math} 
\usepackage[table,dvipsnames]{xcolor}
\usepackage[skip=5pt]{caption}
\definecolor{annals}{RGB}{240,251,247}
\begin{document}
\thispagestyle{empty}

\begin{table}
  \centering
%  \rowcolors{1}{annals!0}{annals!100}
  \setcounter{table}{0} % Set ngives 3
  \caption{A decision tree as a 2x2 table}
  \label{tab:lab}    
  \begin{tabular}{@{}lcc@{}}
    \toprule
    \textbf{Decision Node}&\multicolumn{2}{c}{\textbf{Chance Node}
                            (Evidence)}\\
    \cmidrule(lr){2-3}
    (Hypothesis)  & Event 1 $(E)$ & Event 2 $(E^c)$\\
    \midrule
    Option 1 $(H)$& $ P( E \mid H ) $  & $ P( E^c \mid H) $\\
    Option 2 $(H^c)$& $P(E \mid H^c)$ & $P(E^c \mid H^c) $\\
    \bottomrule
  \end{tabular}
\end{table}


\begin{table}
  \centering
%  \rowcolors{1}{annals!0}{annals!100}
  \setcounter{table}{0} % Set ngives 3
  \caption{A 2-node decision tree as a 2x3 table}
  \label{tab:lab}    
  \begin{tabular}{@{}lccc@{}}
    \toprule
    \textbf{Decision Node}&\multicolumn{2}{c}{\textbf{Chance Node}
                            (Evidence)}& \textbf{Value Node}\\
    \cmidrule(lr){2-3}\cmidrule(l){4-4}
    (Hypothesis)  & Event 1 $(E)$ & Event 2 $(E^c)$ & $v_i =
                                                      f(\text{utility,
                                                      costs, benefits})$\\
    \midrule
    Option 1 $(H)$& $ P( E \mid H ) $  & $ P( E^c \mid H) $ & $v_1(E,
                                                              H)$ \\
    Option 2 $(H^c)$& $P(E \mid H^c)$ & $P(E^c \mid H^c) $ & $v_2(E,
                                                             H^c)$ \\
    \bottomrule
  \end{tabular}
\end{table}
%\begin{align*}
%  P(H \mid E) &= \frac{P(H) P(E \mid H)}{P(E)}\\
%              &= \frac{P(H) P(E \mid H)}
%                { P(H) P(E \mid H) + P(H^c) P(E \mid H^c) }\\
%\end{align*}


%\begin{align*}
%  P(E) &= { P(H) P(E \mid H) + P(H^c) P(E \mid H^c) }\\
%XSCA\end{align*}

\end{document}





%% pdfcrop latex_comicneue_table1_v2.pdf
%%%% Imagemagick conversion
%% magick -density 600 latex_comicneue_table1_v2-crop.pdf -quality 100 latex_comicneue_table1_v2-crop.png


% Local Variables:
% TeX-engine: luatex
% End:
